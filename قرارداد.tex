\documentclass[14pt]{article}
\usepackage[a4paper, margin=2cm]{geometry}
\usepackage[utf8]{inputenc}
\usepackage{xepersian}
\settextfont{Adobe Arabic} % یا یک فونت مناسب دیگر مانند IRKamran، IRNazanin و غیره
%\setdigitfont{Adobe Arabic}
\defpersianfont\titlefont[Scale=1.2]{Adobe Arabic} % فونت برای عنوان اصلی

\title{\titlefont قرارداد اجاره مسکونی/تجاری/اداری}
\author{}
\date{}

\begin{document}
	
	\maketitle
	%\renewcommand{\labelitemi}{$\bullet$}
	
	\section*{ماده 1 – طرفین قرارداد}
	\subsection*{1 – 1) موجر (مالک)}
	آقا/خانم
	\underline{\hspace{3cm}}
	فرزند
	\underline{\hspace{2cm}}
	به شماره شناسنامه
	\underline{\hspace{2cm}}
	صادره از
	\underline{\hspace{1cm}}
	، کد ملی
	\underline{\hspace{2cm}}
	، متولد
	\underline{\hspace{1cm}}
	، ساکن
	\underline{\hspace{4cm}}
	، تلفن ثابت:
	\underline{\hspace{2cm}}
	تلفن همراه:
	\underline{\hspace{2cm}}
	.
	\\
	\textbf{(\textit{در صورت عدم مالکیت مستقیم، یکی از موارد زیر پر شود})}
	
	\begin{itemize}
		\item 
		\textbf{با وکالت}
		از طرف آقا/خانم 
		\underline{\hspace{4cm}}
		فرزند 
		\underline{\hspace{1.5cm}}
		به شماره شناسنامه 
		\underline{\hspace{2cm}}
		، صادره از 
		\underline{\hspace{2cm}}
		، به موجب وکالتنامه شماره 
		\underline{\hspace{2cm}}
		دفترخانه 
		\underline{\hspace{2cm}}
		تاریخ 
		\underline{\hspace{2cm}}
		.
		\item 
		\textbf{با قیمومیت/ولایت}
		برای صغیر/محجور به نام 
		\underline{\hspace{2.5cm}}
		فرزند 
		\underline{\hspace{2cm}}
		به موجب حکم شماره 
		\underline{\hspace{2cm}}
		صادره از دادگاه عمومی 
		\underline{\hspace{2cm}}
		.
		\item 
		\textbf{با وصایت}
		به موجب وصیتنامه رسمی شماره 
		\underline{\hspace{3cm}}
		دفترخانه 
		\underline{\hspace{2cm}}
		.
	\end{itemize}
	
	\subsection*{2 – 1) مستاجر (اجاره‌کننده)}
	آقا/خانم
	\underline{\hspace{3cm}}
	 فرزند
	\underline{\hspace{2cm}}
	 به شماره شناسنامه
	\underline{\hspace{2cm}}
	 صادره از
	\underline{\hspace{1cm}}
	، کد ملی
	\underline{\hspace{2cm}}
	، متولد
	\underline{\hspace{1cm}}
	، ساکن
	\underline{\hspace{4cm}}
	، تلفن ثابت:
	\underline{\hspace{2cm}}
	 تلفن همراه:
	\underline{\hspace{2cm}}
	.
	\\
	\textbf{(\textit{در صورت عدم تصرف مستقیم، یکی از موارد زیر پر شود})}
	\begin{itemize}
		\item 
		\textbf{با وکالت}
		از طرف آقا/خانم 
		\underline{\hspace{4cm}}
		فرزند 
		\underline{\hspace{1.5cm}}
		به شماره شناسنامه 
		\underline{\hspace{2cm}}
		، صادره از 
		\underline{\hspace{2cm}}
		، به موجب وکالتنامه شماره 
		\underline{\hspace{2cm}}
		دفترخانه 
		\underline{\hspace{2cm}}
		تاریخ 
		\underline{\hspace{2cm}}
		.
		\item 
		\textbf{با قیمومیت/ولایت}
		برای صغیر/محجور به نام 
		\underline{\hspace{2.5cm}}
		فرزند 
		\underline{\hspace{2cm}}
		به موجب حکم شماره 
		\underline{\hspace{2cm}}
		صادره از دادگاه عمومی 
		\underline{\hspace{2cm}}
		.
	\end{itemize}
	
	\section*{ماده 2 – موضوع قرارداد و مشخصات مورد اجاره}
	عبارت است از تملیک منافع
	\textbf{یک واحد} (دستگاه/یک باب)
	\textbf{آپارتمان/منزل مسکونی/مغازه تجاری/دفتر کار اداری}
	 واقع در
	\underline{\hspace{5cm}}
	 (نشانی دقیق ملک)، دارای پلاک ثبتی شماره
	\underline{\hspace{2cm}}
	 فرعی از
	\underline{\hspace{2cm}}
	 اصلی، بخش
	\underline{\hspace{2cm}}
	، به مساحت
	\textbf{کل}
	\underline{\hspace{2cm}}
	 متر مربع و
	\textbf{زیربنا}
	\underline{\hspace{2cm}}
	 متر مربع.
	\\
	دارای سند مالکیت به شماره
	\underline{\hspace{3cm}}
	 دفترخانه
	\underline{\hspace{2cm}}
	 شعبه
	\underline{\hspace{1cm}}
	 به نام
	\underline{\hspace{2cm}}
	.
	\\
	مشتمل بر
	\underline{\hspace{1cm}}
	 اتاق خواب،
	\underline{\hspace{1cm}}
	 سرویس بهداشتی، آشپزخانه
	\textbf{(مجهز/غیرمجهز)}
	، با حق استفاده انفرادی از:
	\begin{itemize}
		\item
		\textbf{برق:}
		\textbf{(مستقل/اشتراکی)} با کنتور شماره
		\underline{\hspace{3cm}}
		\item
		\textbf{آب:}
		\textbf{(مستقل/اشتراکی)} با کنتور شماره
		\underline{\hspace{3cm}}
		\item
		\textbf{گاز:}
		\textbf{(مستقل/اشتراکی)} با کنتور شماره
		\underline{\hspace{3cm}}
		\item
		\textbf{سیستم گرمایشی:}
		\textbf{(شوفاژ روشن/پکیج/بخاری)}
		\item
		\textbf{سیستم سرمایشی:}
		\textbf{(کولر آبی/کولر گازی/پنکه سقفی)}
		\item
		\textbf{پارکینگ:}
		\textbf{(فرعی/اشتراکی)} به متراژ
		\underline{\hspace{2cm}} متر مربع
		\item
		\textbf{انباری:}
		\textbf{(فرعی/اشتراکی)} به متراژ
		\underline{\hspace{2cm}} متر مربع
		\item
		\textbf{تلفن:}
		\textbf{(دایر/غیردایر)} به شماره
		\underline{\hspace{3cm}}
		\item
		\textbf{آنتن مرکزی/اینترنت (ISP):}
		\underline{\hspace{4cm}}
	\end{itemize}
	سایر لوازم منصوبات ثابت و مشاعات مربوطه که جهت استفاده به رویت مستاجر رسیده و مورد قبول قرار گرفته است.
	\textbf{(لیست کامل اسباب و اثاثیه، در صورت وجود، در پیوست شماره 1 ذکر شده و جزئی لاینفک این قرارداد است.)}
	
	\section*{ماده 3 – مدت اجاره}
	مدت اجاره
	\underline{\hspace{2cm}} ماه/سال شمسی از تاریخ شروع ۱۴۰۳/\underline{\hspace{.5cm}}/\underline{\hspace{.5cm}} الی تاریخ پایان ۱۴۰۳/\underline{\hspace{.5cm}}/\underline{\hspace{.5cm}} می‌باشد.
	
	\section*{ماده 4 – اجاره بها و نحوه پرداخت}
	\subsection*{4 – 1)}
	میزان اجاره‌بها جمعاً مبلغ
	\underline{\hspace{2.5cm}}
	\textbf{ریال} معادل
	\underline{\hspace{2.5cm}}
	\textbf{تومان}، از قرار ماهیانه مبلغ
	\underline{\hspace{2.5cm}}
	\textbf{ریال} معادل
	\underline{\hspace{2.5cm}}
	\textbf{تومان} که
	\textbf{در روز
	\underline{\hspace{2cm}}
		 (مثلاً: اول) هر ماه} به صورت
	\textbf{نقد/کارت‌به‌کارت/واریز به حساب} به شماره حساب
	\underline{\hspace{3cm}}
	 بانک
	\underline{\hspace{1cm}}
	 به نام
	\underline{\hspace{2cm}}
	 پرداخت می‌شود.
	\subsection*{4 – 2)}
	مبلغ
	\underline{\hspace{2.5cm}}
	\textbf{ریال} معادل
	\underline{\hspace{2.5cm}}
	\textbf{تومان} از طرف مستاجر به عنوان
	\textbf{وجه الضمان (قرض‌الحسنه)} طی
	\textbf{چک} به شماره
	\underline{\hspace{2cm}} بانک
	\underline{\hspace{1cm}} شعبه
	\underline{\hspace{1cm}}
	\textbf{و/or}
	\textbf{نقداً} در تاریخ امضای قرارداد به موجر پرداخت شد. معادل این مبلغ، عیناً و بدون هیچگونه بهره‌ای، همزمان با تخلیهٔ نهایی و عین مستأجره و تسویهٔ کامل کلیه بدهی‌ها (مانند قبوض، شارژ و...) به مستاجر مسترد خواهد شد.
	
	\section*{ماده 5 – تسلیم مورد اجاره}
	موجر مکلف است در تاریخ ۱۴۰۳/\underline{\hspace{.5cm}}/\underline{\hspace{.5cm}} مورد اجاره را مطابق با مشخصات مندرج در ماده (۲) و پیوست آن، سالم و قابل استفاده، با کلیه توابع، ملحقات و منضمات آن جهت استیفاء منفعت به مستاجر تسلیم نماید. رسید تحویل به امضای طرفین خواهد رسید.
	
	\section*{ماده 6 – شرایط و آثار قرارداد}
	\subsection*{6 – 1)}
	مستاجر مجاز است از مورد اجاره
	\textbf{فقط} به منظور
	\textbf{(سکونت/امور تجاری/امور اداری)} استفاده نماید و مکلف است به نحو متعارف و مطابق با عرف از آن محافظت کند.
	\subsection*{6 – 2)}
	مستاجر حق استفاده از مورد اجاره را به نحو مباشرت دارد.
	\textbf{واگذاری حق خود به غیر (اجاره به شرط تملیک، اجارهٔ جزء، واگذاری به دیگری) و همچنین تغییر کاربری بدون کسب مجوز کتبی از موجر ممنوع و موجب فسخ قرارداد است.}
	\subsection*{6 – 3)}
	موجر تأکید می‌نماید که مالک یا متصرف قانونی مورد اجاره بوده و مسئولیت هرگونه دعوای حقوقی و ثبتی اشخاص ثالث نسبت به ملک در طول مدت قرارداد به عهده موجر است.
	\subsection*{6 – 4)}
	در صورت تأخیر مستاجر در پرداخت اجاره‌بها به مدت
	\textbf{بیش از ۱۵ روز} از موعد مقرر، موجر می‌تواند ضمن اخطار کتبی، قرارداد را فسخ و درخواست تخلیهٔ مورد اجاره را از مراجع قضائی صالح بنماید.
	\subsection*{6 – 5)}
	پرداخت کلیه هزینه‌های مصرفی و جاری از قبیل آب، برق، گاز، تلفن، شارژ ساختمان، اینترنت و عوارض نوسازی و فاضلاب، برای مدت زمان تصرف مستاجر، به عهدهٔ وی بوده و باید در موعد تخلیه، رسید پرداخت کلیه قبوض را به موجر ارائه نماید.
	\subsection*{6 – 6)}
	پرداخت هزینه‌های نگهداری و شارژ ماهیانه ساختمان و همچنین افزایش احتمالی آن، بر عهده مستاجر است.
	\subsection*{6 – 7)}
	\textbf{هزینه‌های تعمیرات اساسی و کلی} (از قبیل اساسی ساختمان، نصب موتورخانه، تعمیرات اساسی آسانسور، لوله‌کشی اصلی، نمای ساختمان) به عهده موجر است.
	\textbf{هزینه‌های تعمیرات جزئی و ناشی از استهلاک ناشی از استفاده متعارف} (از قبیل تعمیر شیرآلات، تعویض لامپ، رنگ‌آمیزی درها، تعمیرات جزیی برق و گاز واحد) به عهده مستاجر است.
	\subsection*{6 – 8)}
	پرداخت مالیات بر ملک و عوارض شهرداری مربوط به مالکیت، به عهده موجر و پرداخت مالیات بر درآمد اجاره و مشاغل (در صورت تجاری یا اداری بودن ملک) به عهده مستاجر است.
	\subsection*{6 – 9)}
	\textbf{در خصوص اماکن تجاری:} مبلغ
	\underline{\hspace{3cm}}
	\textbf{ریال} معادل
	\underline{\hspace{3cm}}
	\textbf{تومان} به عنوان
	\textbf{حق کسب و پیشه یا تجارت} توسط مستاجر به موجر
	\textbf{تسلیم و پرداخت گردید/نگردیده است.} (در صورت پرداخت، نحوهٔ محاسبه و استرداد یا انتقال آن در پایان قرارداد باید به طور کامل توضیح داده شود).
	\subsection*{6 – 10)}
	مستاجر مکلف است در زمان تخلیه، مورد اجاره را به همان وضعیت اولیه‌ای که تحویل گرفته (به استثنای استهلاک عادی)، با ارائه رسید قبوض پرداختی، به موجر تحویل داده و رسید اخذ نماید. در صورت ایجاد هرگونه خسارت و نقصان خارج از حد متعارف، مستاجر متعهد به جبران خسارت وارده به نرخ روز خواهد بود.
	\subsection*{6 – 11)}
	موجر ملزم است در زمان تخلیهٔ نهایی و پس از تسویهٔ کامل کلیه بدهی‌های مستاجر، نسبت به استرداد بدون قید و شرط وجه الضمان (قرض‌الحسنه) دریافتی، با اخذ رسید مفصل از مستاجر اقدام نماید.
	\subsection*{6 – 12)}
	در صورتی که موجر نسبت به انجام تعمیرات ضروری که ادامه سکونت یا استفاده را غیرممکن می‌سازد، در مهلت معقولی اقدام نکند، مستاجر می‌تواند با اطلاع قبلی کتبی، شخصاً اقدام لازم را به عمل آورده و هزینه‌های متعارف آن را از اجاره‌بهای بعدی کسر و یا از موجر مطالبه نماید.
	\subsection*{6 – 13)}
	تمدید این قرارداد منحصراً
	\textbf{با توافق کتبی طرفین و قبل از انقضای مدت قرارداد} ممکن خواهد بود. هرگونه توافق برای تمدید، الحاقیه‌ای جداگانه خواهد بود که جزئی لاینفک این قرارداد محسوب می‌شود.
	\subsection*{6 – 14)}
	مستاجر مکلف است به محض اتمام مدت اجاره، عین مستاجره را تخلیه و تسلیم نماید. چنانچه مستاجر در موعد مقرر اقدام به تخلیه ننماید، به ازای هر روز تاخیر، ملزم به پرداخت مبلغ
	\underline{\hspace{3cm}}
	\textbf{ریال} معادل
	\underline{\hspace{3cm}}
	\textbf{تومان} به عنوان
	\textbf{اجرت‌المثل روزانه} به موجر خواهد بود.
	\textbf{تهاتر این خسارت با وجه الضمان (قرض‌الحسنه) بلامانع است.}
	
	\section*{ماده 7 – اسقاط کُلّ خیارات}
	کلیه اختیارات فسخ (شامل خیار غبن، خیار عيب، خیار تاخیر ثمن) به استثنای
	\textbf{خیار تدلیس (فریب)} از طرفین ساقط گردید.
	
	\section*{ماده 8 – حاکمیت قانون}
	این قرارداد در سایر مواردی که در این سند پیش‌بینی نشده است، تابع مقررات قانون مدنی و قانون روابط موجر و مستأجر مصوب ۱۳۷۶/۰۲/۲۲ و اصلاحات بعدی آن خواهد بود.
	
	\section*{ماده 9 – شهود}
	به استناد ماده (۲) قانون روابط موجر و مستأجر مصوب ۱۳۷۶، شهود با مشخصات زیر این قرارداد را امضاء و صحت هویت و رضایت طرفین را گواهی می‌نمایند:
	\begin{itemize}
		\item
		\textbf{شاهد اول:} نام و نام خانوادگی:
		\underline{\hspace{4cm}}، کد ملی:
		\underline{\hspace{3cm}}، شماره تلفن:
		\underline{\hspace{3cm}}
		\item
		\textbf{شاهد دوم:} نام و نام خانوادگی:
		\underline{\hspace{4cm}}، کد ملی:
		\underline{\hspace{3cm}}، شماره تلفن:
		\underline{\hspace{3cm}}
	\end{itemize}
	
	\section*{ماده 10 – حق‌الزحمه مشاور املاک}
	حق‌الزحمه مشاور املاک مطابق تعرفه مصوب اتحادیه، جمعاً مبلغ
	\underline{\hspace{3cm}}
	\textbf{ریال} معادل
	\underline{\hspace{3cm}}
	\textbf{تومان}، به صورت
	\textbf{مساوی (۵۰-۵۰)} به عهده طرفین قرارداد است که همزمان با امضاء، پرداخت و رسید دریافت گردید. فسخ یا اقالهٔ قرارداد به هر دلیل، تاثیری در استرداد حق‌الزحمهٔ وصول شده نخواهد داشت.
	
	\section*{ماده 11 – تعداد نسخ و اعتبار}
	این قرارداد در تاریخ ۱۴۰۳/\underline{\hspace{.5cm}}/\underline{\hspace{.5cm}} در دفتر مشاور املاک
	\underline{\hspace{4cm}} به نشانی
	\underline{\hspace{6cm}} در
	\textbf{سه نسخه} که همه دارای اعتبار واحد می‌باشند، تنظیم، توسط طرفین امضاء و مبادله گردید.
	\begin{itemize}
		\item نسخه اول: به عهده موجر
		\item نسخه دوم: به عهده مستاجر
		\item نسخه سوم: به عهده مشاور املاک (برای بایگانی)
	\end{itemize}
	
	\section*{ماده 12 – توافق خاص}
	کلیه موارد مندرج در این قرارداد مطابق با مقررات جاری کشور بوده و مورد تأیید طرفین است. هرگونه توافق شفاهی فاقد اعتبار است و تنها توافق‌های کتبی مندرج در این قرارداد و پیوست‌های آن معتبر خواهند بود.
	
	\section*{پیوست ها:}
	\begin{enumerate}
		\item صورتجلسه تحویل ملک و لیست کامل اثاثیه (در صورت وجود)
		\item کپی مصدق سند مالکیت و کارت ملی موجر
		\item کپی مصدق کارت ملی مستاجر
		\item کپی چک‌های پرداختی (در صورت وجود)
	\end{enumerate}
	
	\vspace{2cm}
	\centerline{\textbf{--- امضای طرفین و شهود: ---}}
	
	\vspace{1cm}
	\noindent
	\textbf{موجر:}
	\hrulefill
	\\
	\vspace{0.5cm}
	\textbf{مستاجر:}
	\hrulefill
	\\
	\vspace{0.5cm}
	\textbf{شاهد اول:}
	\hrulefill
	\\
	\vspace{0.5cm}
	\textbf{شاهد دوم:}
	\hrulefill
	\\
	\vspace{0.5cm}
	\textbf{مشاور املاک (مُبَرز):}
	\hrulefill
	\hspace{1cm} (مهر و امضاء)
	
	\vspace{2cm}
	\section*{\textbf{نکات مهم برای استفاده:}}
	\begin{enumerate}
		\item قبل از امضاء، تمام صفحات قرارداد را با دقت بخوانید.
		\item از صحت تمام اطلاعات وارد شده (شماره حساب، کد ملی، مشخصات ملک) اطمینان حاصل کنید.
		\item حتماً از چک‌های دریافتی (به ویژه چک وجه الضمان) رسید بگیرید.
		\item
		\textbf{صورتجلسه تحویل ملک} در ابتدا و انتهای قرارداد بسیار مهم است. از وضعیت ملک عکس و فیلم بگیرید و آن را به امضای دو طرف برسانید.
		\item در صورت امکان، تنظیم قرارداد را نزد یک
		\textbf{دفترخانه اسناد رسمی} انجام دهید تا از اعتبار بیشتر و جلوگیری از مشکلات آینده برخوردار شوید.
	\end{enumerate}
	
\end{document}