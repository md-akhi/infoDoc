\documentclass[12pt]{article}
\usepackage[a4paper, margin=2cm]{geometry}
\usepackage[utf8]{inputenc}
\usepackage{xepersian}
\settextfont{Adobe Arabic} % یا یک فونت مناسب دیگر
%\setdigitfont{Adobe Arabic}

\title{\textbf{پیوست شماره 1 - صورتجلسه تحویل و لیست اثاثیه}}
\author{}
\date{}

\begin{document}
	
	\maketitle
	
	\section*{\textbf{صورتجلسه تحویل و عودت ملک}}
	\subsection*{مشخصات قرارداد اصلی}
	\begin{itemize}
		\item \textbf{شماره قرارداد:} ................
		\item \textbf{تاریخ قرارداد:} ۱۴۰۳/..../....
		\item \textbf{موجر (مالک):} ................
		\item \textbf{مستاجر:} ................
		\item \textbf{آدرس ملک:} ................
	\end{itemize}
	
	\subsection*{الف) صورتجلسه تحویل (ابتدای قرارداد - تاریخ: ۱۴۰۳/..../....)}
	\noindent
	بدینوسیله مُقرّر می‌گردد، مورد اجاره مطابق با مشخصات مندرج در قرارداد فوق، در تاریخ فوق‌الذکر توسط موجر به مستاجر تحویل و توسط مستاجر تحویل گرفته شد.
	
	\noindent
	\textbf{وضعیت ملک در زمان تحویل:}
	
	\begin{itemize}
		\item کلیه انشعابات (برق، آب، گاز، تلفن) در وضعیت \textbf{(فعال/غیرفعال)} و سالم تحویل گردید.
		\item سیستم‌های گرمایشی و سرمایشی در وضعیت \textbf{(سالم/معیوب)} تحویل گردید.
		\item کلیه درها، پنجره‌ها، شیشه‌ها، قفل و کلیدها سالم و قابل استفاده می‌باشد.
		\item رنگ دیوارها، کفپوش و سقف در وضعیت \textbf{(نو/فرسوده معمولی/نیازمند تعمیر)} تحویل گردید.
		\item سرویس‌های بهداشتی و آشپزخانه در وضعیت \textbf{(سالم/معیوب)} تحویل گردید.
	\end{itemize}
	
	\noindent
	\textbf{تذکر مهم:} مستاجر متعهد می‌گردد در زمان تخلیه، ملک را به همان حالت اولیه (به استثنای استهلاک عادی ناشی از استفاده متعارف) عودت دهد.
	
	\vspace{1cm}
	\noindent
	\textbf{امضای موجر:} \hrulefill \hspace{2cm} \textbf{امضای مستاجر:} \hrulefill
	
	\vspace{2cm}
	\subsection*{ب) لیست کامل اثاثیه و لوازم منصوب (در صورت وجود)}
	\noindent
	\textbf{تذکر:} این لیست تنها در صورتی تکمیل می‌شود که ملک به صورت \textbf{مُجهّز} اجاره داده شود. در غیر این صورت، بندهای زیر حذف می‌شوند.
	
	\begin{center}
		\renewcommand{\arraystretch}{1.5}
		\begin{tabular}{|c|c|c|c|c|}
			\hline
			\textbf{ردیف} & \textbf{نام قلم کالا} & \textbf{تعداد} & \textbf{شرایط و توضیحات (نو، کهنه، سالم، معیوب)} & \textbf{برآورد قیمت (ریال)} \\
			\hline
			1 & یخچال فریزر & 1 & سالم، مدل ۱۴۰۰ & ۵۰,۰۰۰,۰۰۰ \\
			\hline
			2 & اجاق گاز & 1 & سالم، چهار شعله & ۱۵,۰۰۰,۰۰۰ \\
			\hline
			3 & هود & 1 & سالم & ۵,۰۰۰,۰۰۰ \\
			\hline
			4 & ماشین لباسشویی & 1 & معیوب، نیاز به تعمیر & ۱۰,۰۰۰,۰۰۰ \\
			\hline
			5 & تلویزیون & 1 & سالم، ۴۳ اینچ & ۴۰,۰۰۰,۰۰۰ \\
			\hline
			6 & کولر گازی & 1 & سالم، پرتابل & ۳۰,۰۰۰,۰۰۰ \\
			\hline
			7 & مبل راحتی & 1 ست & فرسوده، پارچه پاره & ۲۰,۰۰۰,۰۰۰ \\
			\hline
			8 & فرش & 2 عدد & سالم، ۶ متر & ۶۰,۰۰۰,۰۰۰ \\
			\hline
			9 & کمد دیواری & 1 عدد & سالم، سه لنگه & ۲۵,۰۰۰,۰۰۰ \\
			\hline
			10 & تختخواب & 2 عدد & سالم، دونفره & ۳۵,۰۰۰,۰۰۰ \\
			\hline
			& \textbf{جمع کل برآورد:} & & & \textbf{۲۹۰,۰۰۰,۰۰۰ ریال} \\
			\hline
		\end{tabular}
	\end{center}
	
	\vspace{1cm}
	\noindent
	\textbf{تعهد مستاجر:} مستاجر متعهد می‌گردد کلیه اسباب و اثاثیه فوق را به نحو مطلوب نگهداری نموده و در پایان مدت اجاره، به همان کیفیت اولیه (به استثنای استهلاک عادی) عودت دهد. در صورت ایجاد خسارت، تعمیر یا جایگزینی آن به عهده مستاجر خواهد بود.
	
	\vspace{1cm}
	\noindent
	\textbf{امضای موجر:} \hrulefill \hspace{2cm} \textbf{امضای مستاجر:} \hrulefill
	
	\vspace{2cm}
	\subsection*{ج) صورتجلسه عودت (پایان قرارداد - تاریخ: ۱۴۰۳/..../....)}
	\noindent
	بدینوسیله مُقرّر می‌گردد، مورد اجاره مطابق با مشخصات مندرج در قرارداد، در تاریخ فوق‌الذکر توسط مستاجر به موجر عودت و توسط موجر تحویل گرفته شد.
	
	\noindent
	\textbf{وضعیت ملک در زمان عودت:}
	
	\begin{itemize}
		\item کلیه بدهی‌های انشعابات (قبض آب، برق، گاز، تلفن) تسویه شده است. (رسید پیوست است)
		\item هزینه‌های شارژ ساختمان تا تاریخ تخلیه تسویه شده است. (رسید پیوست است)
		\item وضعیت ملک و اثاثیه \textbf{(مطابق با زمان تحویل/دارای آسیب‌های زیر)} می‌باشد.
	\end{itemize}
	
	\noindent
	\textbf{شرح آسیب‌ها و خسارات وارده (در صورت وجود):}
	\vspace{1cm}
	\hrule
	\vspace{0.5cm}
	\hrule
	\vspace{0.5cm}
	\hrule
	
	\noindent
	\textbf{تسویه حساب:}
	\begin{itemize}
		\item مبلغ وجه الضمان (قرض‌الحسنه): \underline{\hspace{4cm}} ریال
		\item کسورات (خسارات، بدهی‌ها): \underline{\hspace{4cm}} ریال
		\item مبلغ قابل برگشت به مستاجر: \underline{\hspace{4cm}} ریال
	\end{itemize}
	
	\noindent
	بدینوسیله رسیدگی و تسویه کامل گردید و هیچگونه ادعایی برای هیچ یک از طرفین باقی نماند.
	
	\vspace{1.5cm}
	\noindent
	\textbf{امضای موجر:} \hrulefill \hspace{2cm} \textbf{امضای مستاجر:} \hrulefill
	
\end{document}